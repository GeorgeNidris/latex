\section{Ορισμός μεταβλητών σχεδιασμού}

Για το δεδομένο πρόβλημα, οι μεταβλητές σχεδιασμού είναι τέσσερις συντελεστές που ελέγχουν την κατανομή της κατακόρυφης ταχύτητας που εκχέεται ή απορροφάται απο την πλάκα. Συγκεκριμένα, η κατανομή της κατακόρυφης ταχύτητας $v_0$ ελέγχεται με βάση τη σχέση \ref{eq:vvel}.

\begin{equation}
v_0(x) = ax^3 + b x^2 + cx + d
\label{eq:vvel}
\end{equation}

\vspace{10pt}

Ωστόσο, έχουμε περιορισμό στην συνολική παροχή, τον οποίο πρέπει να τηρεί η κατανομή της ταχύτητας. Όπως αναφέραμε, ανά μονάδα μήκους η παροχή είναι:

\begin{equation}
    \begin{aligned}
        Q =& \int_0^L(ax^3+bx^2+cx+d)dx\\[8pt]
        Q =& \Big[\dfrac{a}{4}x^4 + \dfrac{b}{3}x^3 + \dfrac{c}{2}x^2 + d\cdot x\Big]^L_0\\[8pt]
        Q =& \dfrac{a}{4}L^4 + \dfrac{b}{3}L^3 + \dfrac{c}{2}L^2 + dL
    \end{aligned}
    \label{eq:paroxi}
\end{equation}

 \vspace{10pt}

 Επομένως, μια απο τις τέσσερις μεταβλητές σχεδιασμού πρέπει να τεθεί ως εξαρτημένη και να τηρεί την παραπάνω σχέση ικανοποιώντας τον περιορισμό για σταθερή παροχή. 

 Έτσι, εκφράζουμε την τέταρτη μεταβλητή $d$ ώς:

 \begin{equation}
 d = \dfrac{Q - \big(\dfrac{a}{4}L^4 + \dfrac{b}{3}L^3 + \dfrac{c}{2}L^2 \big)}{L} 
    \label{eq:deltaFin}
 \end{equation}

 \vspace{10pt}

 Επομένως, πλέον, οι μεταβλητές σχεδιασμού μας είναι οι συντελεστές $a,b,c$ του πολυωνύμου που εκφράζει την κατανομή της κατακόρυφης ταχύτητας $v_0$ της ροής που εκχέει - απορροφά η πλάκα.

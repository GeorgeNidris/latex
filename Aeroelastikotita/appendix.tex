\section{Πηγαίος κώδικας}

Το λογισμικό που λύνει το μοντέλο υλοποιήθηκε σε γλώσσα C++. Η δομή του κώδικα αποτελείται απο τα εξής. Ενα αρχείο (main.cpp) με την κύρια συνάρτηση του προγράμματος που περιλαμβάνει την κεντρική δομή του αλγορίθμου, και δευτερεύοντα βοηθητικά αρχεία που περιέχουν τους ορισμούς των συναρτήσεων που καλούνται στην κεντρική δομή του αλγορίθμου. Οι παράμετροι του προβλήματος και τα μητρώα μάζας και δυσκαμψίας διαβάζονται απο αντίστοιχα αρχεία εισόδου.
Έτσι, παρακάτω παρατίθενται τα αρχεία με την σειρά που αναφέρθηκαν.

\subsection{Κύρια συνάρτηση αλγορίθμου main.cpp}

\lstinputlisting[]{source/main.cpp}

\vspace{4.5cm}
\subsection{Βοηθητικό αρχείο συναρτήσεων utilities.cpp}

\lstinputlisting[]{./source/utilities.cpp}

\vspace{3cm}
\subsection{Αρχείο συναρτήσεων επίλυσης solver.cpp}

\lstinputlisting[]{./source/solver.cpp}

\vspace{3cm}
\subsection{Αρχείο συναρτήσεων εξισώσεων μόνιμης ροής steady.cpp}

\lstinputlisting[]{./source/steady.cpp}
\vspace{3cm}
\subsection{Αρχείο συναρτήσεων εξισώσεων μη-μόνιμης ροής theodorsen.cpp}

\lstinputlisting[]{./source/theodorsen.cpp}

\vspace{3cm}
\subsection{Βοηθητικό αρχείο ορισμού συναρτήσεων utilities.h}

\lstinputlisting[]{./source/utilities.h}

\vspace{3cm}
\subsection{Αρχεία εισόδου matrix.dat και input.dat}

\label{AppLastPage}
\lstinputlisting[]{./source/input.dat}
\vspace{3cm}
\lstinputlisting[]{./source/matrix.dat}


\section{Πηγαίος κώδικας}

Το λογισμικό που λύνει το μοντέλο υλοποιήθηκε σε γλώσσα C++. Η δομή του κώδικα αποτελείται απο τα εξής. Ενα κύριο αρχείο (main.cpp) με την κύρια συνάρτηση του προγράμματος που περιλαμβάνει την κεντρική δομή του αλγορίθμου, και ένα δευτερεύον βοηθητικό αρχείο (utils.cpp) που περιέχει τους ορισμούς των συναρτήσεων που καλούνται στην κεντρική δομή του αλγορίθμου. Τέλος, έχουμε ένα αρχείο που περιέχει τις γεωμετρικές παραμέτρους του προβλήματος και τις αριθμητικές παραμέτρους του κώδικα. Η δομή του main.cpp ακολουθεί τη ροή του αλγορίθμου που παρουσιάζεται στο σχήμα \ref{fig:flowchart}.
Έτσι, παρακάτω παρατίθενται τα αρχεία με την σειρά που αναφέρθηκαν.

\subsection{Κύρια συνάρτηση αλγορίθμου main.cpp}

%\lstinputlisting[]{source/main.cpp}

\vspace{4.5cm}
\subsection{Βοηθητικό αρχείο συναρτήσεων utils.cpp}

%\lstinputlisting[]{source/utils.cpp}

\vspace{3cm}
\subsection{Αρχείο παραμέτρων data.h}
%\lstinputlisting[]{source/data.h}

\section{Χρονική ολοκλήρωση των αεροελαστικών εξισώσεων}

Η χρονική απόκριση των δυο βαθμών ελευθερίας εκφράζονται ως εξής:

\begin{equation}
    \mathbf{x}(t) = \mathbf{x}_{hom} + \mathbf{x}_{part}
    \label{eq:timesol}
\end{equation}

Όπου $\mathbf{x}_{hom}$ η ομογενής λύση του συστήματος και $\mathbf{x}_{part}$ η μερική λύση που ακολουθεί τη μορφή της διέγερσης -- των αεροδυναμικών φορτίων.

Η ομογενής λύση, χρησιμοποιώντας τον ιδιοανυσματικό μετασχηματισμό είναι:

\begin{equation}
    \mathbf{x}(t)_{hom} = \sum_{k=1}^2\Bigg[\phi_{0k}e^{Re(\lambda_k)}\Big(A_kcos(Im(\lambda_k)t+\theta_k)+B_ksin(Im(\lambda_k)t+\theta_k)\Big)\Bigg]
    \label{eq:homogenous}
\end{equation}

Όπου, με k συμβολίζουμε την k-oστή ιδιοτιμή, $\phi_{0k}$ είναι το μέτρο της μιγαδικής ιδιομορφής και $\theta_k$ η φάση της. Δηλαδή, η ομογενής λύση προκύπτει απο υπέρθεση των ταλαντώσεων για όλες τις ιδιοτιμές. Οι συντελεστές $A_k, B_k$ προσδιορίζονται απο τις αρχικές συνθήκες. 

Η αντίστοιχη μερική λύση, θεωρώντας σταθερά αεροδυναμικά φορτία είναι η μόνιμη απόκριση του συστήματος και είναι:

\begin{equation}
    \mathbf{x}_{part} = \begin{bmatrix}
    \mathbf{0} & \mathbf{0}\\
    \mathbf{0} & \mathbf{K}^{-1}
    \end{bmatrix}
    \cdot
    \begin{Bmatrix}
    0\\0\\F_x\\F_z
    \end{Bmatrix}
    \label{eq:partial}
\end{equation}

Επομένως, με δεδομένα τα παραπάνω, μπορούμε να προσδιορίσουμε την απόκριση του συστήματος. Για την περίπτωση της μόνιμης ροής, γραμμικοποιήσαμε γύρω απο τη θέση ισορροπίας, και χρησιμοποιήσαμε τις παραπάνω εξισώσεις για τον υπολογισμό της χρονοσειράς της απόκρισης, με απόσβεση και σταθερό αεροδυναμικό φορτίο όπως υπολογίζεται στη θέση αναφοράς.

Για την περίπτωση της μη-μονιμης ροής, όπως αναφέραμε και σε προηγούμενη παράγραφο, επειδή για τον υπολογισμό των αεροδυναμικών φορτίων απαιτείται η επίλυση μιας ακόμη διαφορικής εξίσωσης, σε κάθε χρονικό βήμα, με αρχική συνθήκη την κατάσταση του προηγούμενου βήματος, υπολογίζουμε τα αεροδυναμικά φορτία, και προσδιορίζουμε τη νεα κατάσταση απο τις \crefrange{eq:timesol}{eq:partial}. Εδώ κάνουμε την απλούστευση πως για τη διάρκεια κάθε βήματος, τα αεροδυναμικά φορτία είναι σταθερά και υπολογίζουμε τη μερική λύση απο την \cref{eq:partial}. Επιπλέον, σε αυτή την περίπτωση, στην ομογενή λύση χρησιμοποιούμε μηδενικό μητρώο απόσβεσης, αφού η απόσβεση υπεισέρχεται έμμεσα απο την μεταβολή των αεροδυναμικών φορτίων ανάλογα της κατάστασης του συστήματος. 

Ξεκινώντας με μηδενικές μετατοπίσεις, ταχύτητες, και επιταχύνσεις, και χρησιμοποιώντας τα δεδομένα όπως παρατίθενται στον πίνακα \ref{tab:data}, η χρονοσειρά της απόκρισης για μόνιμη και μη-μόνιμη ροή, φαίνονται στα \crefrange{fig:responseu}{fig:responsew}.

\begin{figure}
    \begin{center}
        \includegraphics[width=0.95\textwidth]{figures/}
    \end{center}
    \caption{}\label{fig:}
\end{figure}




\section{Κατασκευή πλέγματος}

Όπως αναφέρθηκε και προηγουμένως, το πρόβλημα μοντελοποιείται ως αξονοσυμμετρικό επιλύεται σε διδιάστατο πλέγμα. Για την αυξηση της ακρίβειας των αριθμητικών μεθόδων, εφαρμόσθηκε πύκνωση στις περιοχές όπου έχουμε υψηλές κλίσεις και συγκεκριμένα, στην περιοχή γύρω απο τον δρομέα, στο τοίχωμα της αεροσύραγγας, και κοντά στον άξονα συμμετρίας όπου αναπτύσσεται ο ομόρρου της ανεμογεννήτριας. 

\subsection{Σχηματισμός πλέγματος κατά την ακτινική διεύθυνση}

Κατά την ακτινική διεύθυνση το πλέγμα χωρίστηκε σε τρείς περιοχές. Αρχικά στην περιοχή απο τον άξονα συμμετρίας έως την ακτίνα του δρομέα έχουμε ομοιόμορφο πλέγμα με μέγεθος {dy\_grid} που εισάγεται απο τον χρήστη. Έπειτα, το υπόλοιπο ακτινικό χωρίο χωρίζεται στα δύο. Απο τη θέση y=1 (ακτίνα δρομέα) έως το μέσο του υπολοιπόμενου χωρίου, έχουμε σταδιακή αραίωση με γεωμετρική πρόοδο. Ενώ απο το μέσο του χωρίου έως το τοίχωμα έχουμε σταδιακή πύκνωση με τον ίδιο λόγο γεωμετρικής προόδου, επομένως έχουμε αντικατωπτρισμό του πλέγματος που δημιουργήθηκε στην δεύτερη περιοχή. Το πλέγμα σε αυτές τις περιοχές ελέγχεται απο τον χρήστη εισάγωντας τον αριθμό των κελλιών σε κάθε περιοχή και τον λόγο της γεωμετρικής προόδου. Οι τρεις περιοχές και το παραγόμενο πλέγμα φαίνονται στο σχήμα \ref{fig:ygrid}.

\newpage
\begin{figure}[thp]
\centering
\begin{tabular}{c}
    \begin{lstlisting}[label={lst:code}, mathescape=true, breaklines=true, linewidth=.6\textwidth]
!--- Uniform grid in y-direction 
    !From y=ymin to y=1 (disk radius)

! Second point is on symmetry line
y_grid(2)=ymin 

do j=2,ngridy1-1
    y_grid(j+1)=y_grid(j) + dy_grid
enddo

! First point is symmetric to third
y_grid(1)=-y_grid(3) 

    \end{lstlisting}
\end{tabular}
\caption*{Test}
\end{figure}

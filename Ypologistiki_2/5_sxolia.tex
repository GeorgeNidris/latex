\section{Σχολιασμός αποτελεσμάτων}

Ξεκινώντας, παρατηρούμε πως το μοντέλο μπόρεσε να αντικατοπτρίσει την ανάπτυξη του οριακού στρώματος στο τοίχωμα της σύραγγας ικανοποιητικά. Επιπλέον, η μοντελοποίηση του δρομέα με τη χρήση της θεωρίας δίσκου ορμής, όπως φαίνεται και στα σχήματα \ref{fig:isoU_zoom} και \ref{fig:x0prof} δρά σωστά ως καταβόθρα ορμής και δημιουργεί ενα σκαλοπάτι στην τιμή της οριζόντιας ταχύτητας u. Επιπλέον, απο την επίλυση του πεδίου παρατηρούμε επίσης την επιτάχυνση της ροής γύρω απο τον δρομέα, και για την ικανοποίηση της εξίσωσης της συνέχειας, έχουμε αύξηση της κατακόρυφης ταχύτητας v και την επιτάχυνση της ροής εντός του οριακού στρώματος. Δηλαδή, η παρουσία του δρομέα, ωθεί την ροή να κινηθεί γύρω του.

Περεταίρω αναφορικά με τη μοντελοποίηση του δρομέα, προφανώς το συγκεκριμένο μοντέλο δεν επιλύει τα χαρακτηριστικά της ώσης του δρομέα, παραμόνο την ροή γύρω απο αυτόν με δεδομένη τη συμπεριφορά του δρομέα απο τα δεδομένα της εκφώνησης. Δηλαδή, αν θα θέλαμε να υπολογίσουμε μόνοι μας την επίδραση της ροής στον δρομέα, δηλαδή να εξάγουμε κάποια χαρακτηριστική καμπύλη ώσης (ή συντελεστή ώσης) συναρτήσει της ταχύτητας, θα έπρεπε να τον μοντελοποιήσουμε ως στερεό σώμα, με τα γεωμετρικά χαρακτηριστικά του και την αντίστοιχη γωνιακή του ταχύτητα, και φυσικά σε αυτή την περίπτωση, το μοντέλο δεν θα μπορούσε να περιγραφεί ως αξονοσυμμετρικό και θα χρειαζόμασταν ενα τρισδιάστατο πλέγμα με τρισδιάστατη έκφραση των εξισώσεων Navier-Stokes. 

\subsection{Επίδραση τοιχωμάτων στην τιμή της ώσης}
Το παρόν μοντέλο είναι ικανό να υπολογίσει την επίδραση των χαρακτηριστικών της σύραγγας και του δρομέα στην τιμή της ώσης. Πράγματι, μπορούμε να εκτιμήσουμε την επίδραση της σύραγγας (των τοιχωμάτων δηλαδή, και συνεπώς του οριακού στρώματος) στην τιμή της ώσης για ίδια ταχύτητα ελεύθερης ροής.

Λόγω της ανάπτυξης του οριακού στρώματος, ακολουθώντας την εξίσωση της συνέχειας και διατήρησης ορμής, η ταχύτητα κοντά στο κέντρο της σύραγγας επιταχύνεται ώστε η συνολική παροχή σε ολόκληρη τη διατομή της σύραγγας να είναι ίση με την παροχή στην είσοδο της σύραγγας με ομοιόμορφο πεδίο ταχύτητας. 

Έτσι, ακολουθώντας τα δεδομένα της εκφώνησης για τον συντελεστή ώσης έχουμε:

\begin{table}[h!]
    \begin{center}
        \begin{tabular}[c]{|r|c|c|}
            \hline
            & Σύραγγα & Ελεύθερη ροή\\
            \hline
            $U_{ref} (m/s)$ & 17.67 & 14 \\
            $C_t$ & 0.1266 & 0.26 \\
            $T (N)$ & 8.78 & 11.32 \\
            \hline
            Ποσοστιαία μεταβολή  & -22.43\% &  \\
            \hline
        \end{tabular}
    \end{center}
    \caption{Επίδραση σύραγγας στην ώση}
    \label{tab:thrust}
\end{table}

Παρατηρούμε πως ενώ θα αναμέναμε την τιμή της ώσης να αυξηθεί λόγω της αυξημένης ταχύτητας, η ταχύτητα της ελεύθερης ροής μας έιναι 14 m/s που έιναι ήδη υψηλή για ανεμογεννήτρια. Μάλιστα, ήδη απο τα 12 m/s περίπου είτε λόγω αποκόλλησης στα blades της Α/Γ είτε μέσω ενεργητικού ελέγχου, μειώνεται σκόπιμα ο συντελεστής ώσης, ώστε να προστατεύσει την εγκατάσταση, το δίκτυο κ.α. Έτσι, παρ'όλο που η επίδραση του τοιχώματος αυξάνει την τιμή της ταχύτητας στην θέση του δρομέα, ο συντελεστής ώσης πέφτει απότομα και ώς συνολική επίδραση η ώση μειώνεται. 

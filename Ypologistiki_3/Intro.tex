\section*{Εισαγωγή}
\addcontentsline{toc}{section}{Εισαγωγή}

Η παρούσα εργασία στοχεύει στην ανάπτυξη λογισμικού που λύνει τις μονοδιάστατες εξισώσεις Euler για μοντελοποίηση της ροής συμπιεστής ροής εντός αγωγού μεταβλητής διατομής. Θεωρούμε πως στο αριστερό μέρος του αγωγού έχουμε συνδεδεμένο αεροφυλάκιο σταθερών και γνωστών συνθηκων. Η ροή μελετάται ως μη μόνιμη και παρουσιάζεται το μεταβατικό της στάδιο απο την εκκίνησή της ανοίγωντας το αεροφυλάκιο έως την αποκατάσταση της μόνιμης ροής. 

Οι εξισώσεις Euler παρουσιάζονται στην πρωταρχική τους μορφή (\ref{eq:prim}) και την συντηρητική τους μορφή (\ref{eq:cons}) που βοηθά στην επίλυσή τους. Η πρωταρχική μορφή μας παρέχει τις ιδιοτιμές της Ιακωβιανής οι οποίες είναι αναγκαίες για να αναγνωρίσουμε τη ροή της πληροφορίας και τον υπολογισμό των παροχών όπως θα δούμε παρακάτω.

\begin{align}
    \dpart{}{t}\begin{pmatrix}
    \rho\\u\\p
    \end{pmatrix} + 
    \begin{pmatrix}
    u & \rho & 0\\
    0 & u & 1/\rho\\
    0 & \rho c^2 & u
    \end{pmatrix}
    \cdot \dpart{}{x}
    \begin{pmatrix}
    \rho\\u\\p
    \end{pmatrix}
    =&
    \begin{pmatrix}
    -\dfrac{\rho u}{S}\dfrac{dS}{dx}\\
    0\\
    -\dfrac{\rho u c^2}{S}\dfrac{dS}{dx}\\
    \end{pmatrix}
    %
    &&\text{\textbf{Πρωταρχική μορφή}}
    \label{eq:prim}\\
    %
    %--------------------
    %
    \dpart{}{t}\begin{pmatrix}
    \tilde{\rho}\\
    \tilde{\rho}u\\
    \tilde{\rho}E
    \end{pmatrix} + 
    \dpart{}{x}
    \begin{pmatrix}
    \tilde{\rho}u\\
    \tilde{\rho}u^2+\tilde{p}\\
    \tilde{\rho}Hu\\
    \end{pmatrix}
    =&
    \begin{pmatrix}
    0\\
    \dfrac{\tilde{p}}{S}\dfrac{dS}{dx}\\
    0\\
    \end{pmatrix} \lmath
    %
    &&\text{\textbf{Συντηρητική μορφή}}%eksigisi
    \label{eq:cons}
\end{align}


Όπου,

\begin{itemize}
    \item $\mathbf{\widetilde{\rho} =\rho S}$
    \item $\mathbf{\tilde{p} = pS}$
\end{itemize}

Όπως αναφέραμε και παραπάνω, θα λύσουμε την συντηρητική μορφή της εξίσωσης του Euler που επιτρέπει την άμεση ολοκλήρωσή της, και η μεθοδολογία που ακολουθήσαμε για την επίλυση αναλύεται στην παρακάτω παράγραφο.

\section{Μεθοδολογία επίλυσης}

Ολοκληρώνωντας την εξίσωση \ref{eq:cons} και γράφοντάς τη σε διακριτή μορφή (προσαρμόζοντας στο πλέγμα) αποκτάμε την παρακάτω μορφή (εξ \ref{eq:discrete}).

\begin{equation}
    \Delta x\dpart{\overline{U}_i}{t} + (F_{i+1/2} - F_{i-1/2}) = \Delta x \cdot \overline{Q}_i
    \label{eq:discrete}
\end{equation}

Όπου $\overline{U}_i \text{ και } \overline{Q}_i$ είναι η μέση τιμή των συντηρητικών μεταβλητών και των όρων πηγής αντίστοιχα σε ένα κελλί. Επομένως, η κεντρική ιδέα της μεθόδου είναι πως χρησιμοποιώντας την διακριτοποιημένη έκφραση της \ref{eq:discrete}, εκτελούμε σχήμα ρητής ολοκλήρωσης Runge Kutta 4ης τάξης για τον υπολογισμό της χρονικής εξέλιξης της ροής (παράγραφος \ref{sub:rk}). 

Οι τιμές των παροχών στα σύνορα (faces) των κελλιών ($F_{i+1/2}, F_{i-1/2}$), υπολογίζονται με τη βοήθεια του σχήματος ανακατασκευής του Roe (παράγραφος \ref{sub:roe}).

\subsection{Σχήμα ανακατασκευής του Roe} \label{sub:roe} 

Το σχήμα του Roe υπολογίζει τις παροχές στις επιφάνειες των κελλιών μέσω της σχέσης \ref{eq:roe}. Πρακτικά υπολογίζει τον μέσο όρο των παροχών που προκύπτουν απο το αριστερό και το δεξί γειτονικό κελλι, και διορθώνει την τιμή λαμβάνωντας υπόψη, μέσω των ιδιοτιμών, την κατεύθυνση που μεταφέρεται η πληροφορία. 

\begin{equation}
    F_f = \half (F_L + F_R) - \half |A|\cdot (U_R-U_L) 
    \label{eq:roe}
\end{equation}

Όπου,

\begin{itemize}
    \item $\mathbf{F_L, F_R}$, οι τιμές της παροχής ($\tilde{\rho}u, \tilde{\rho}u^2+\tilde{p}, \tilde{\rho}Hu$) χρησιμοποιώντας τη μέση τιμή του αντίστοιχου συντηρητικού μεγέθους (σχήμα 1ης τάξης) για το αντίστοιχο κελί (αριστερό ή δεξί)
    \item $\mathbf{|A|=R|Λ|R^{-1}}$, με $|\Lambda|$ τον διαγώνιο πίνακα που έχει τις απόλυτες τιμές των ιδιοτιμών στη διαγώνιό του 
    \item $\mathbf{U_R-U_L}$, το διάνυσμα της διαφοράς των συντηρητικών μεταβλητών του δεξιού μείον του αριστερού κελλιού
\end{itemize}


Για τον υπολογισμό του μητρώου $|A|$ χρησιμοποιούμε μια έκφραση των τιμών που αποτελεί ένα σταθμισμένο μέσο όρο των τιμών του αριστερού και του δεξιού κελλιού (εξ \ref{eq:roeVars}).

\begin{equation}
   \begin{aligned}
     \tilde{\rho} &= \sqrt{\rho_L}\cdot\sqrt{\rho_R}\\
     \tilde{u} &= \dfrac{\sqrt{\rho_L}\cdot u_L+\sqrt{\rho_R}\cdot u_R}{\sqrt{\rho_L}+\sqrt{\rho_R}}\\
     \tilde{H} &= \dfrac{\sqrt{\rho_L}\cdot H_L+\sqrt{\rho_R}\cdot H_R}{\sqrt{\rho_L}+\sqrt{\rho_R}}\\
     \tilde{c} &= \sqrt{(\gamma-1)(\tilde{H}-0.5\tilde{u}^2)}
   \end{aligned} 
    \label{eq:roeVars}
\end{equation}

Με $H = E + pv = E + \dfrac{p}{\rho} = \dfrac{\dfrac{\tilde{\rho} E}{S} + p}{\rho}$
\vspace{6pt}


Οπότε, με βάση τα μεγέθη των σχέσεων \ref{eq:roeVars}, οι ιδιοτιμές και το μητρώο $R$ είναι:

\begin{equation}
   \begin{aligned}
        |\lambda| &= \begin{bmatrix}
            |\tilde{u}| & 0 & 0\\
            0 & |\tilde{u} + \tilde{c}| & 0\\
            0 & 0 & |\tilde{u} - \tilde{c}|
        \end{bmatrix}\\
        R &= \begin{bmatrix}
            1 & 0.5\rb/\cb & -0.5\rb/\cb\\
            \ub & 0.5(\ub+\cb)\rb/\cb & -0.5(\ub-\cb)\rb/\cb\\
            0.5\ub^2 & 0.5\cdot(0.5\ub^2+\ub\cb+\dfrac{\cb^2}{\gamma-1})\rb/\cb & -0.5\cdot(0.5\ub^2-\ub\cb+\dfrac{\cb^2}{\gamma-1})\rb/\cb\\
            \end{bmatrix}
   \end{aligned} 
    \label{eq:roeMat}
\end{equation}

Τέλος, για την αντιμετώπιση ασταθειών στην περίπτωση κρουστικού κύματος, διορθώνουμε τις ιδιοτιμές ως εξής:

\begin{align*}
    \delta &= 0.05\cdot \cb^2\\
    |\lambda| &= \dfrac{|\lambda|^2+\delta^2}{2\delta}  \text{ εαν } |\lambda| \leq \delta
\end{align*}

Έτσι, μέσω των παραπάνω, υπολογίζουμε τις παροχές σε κάθε επιφάνεια του πλέγματός μας. Το αντίστοιχο τμήμα του κώδικα που πραγματοποιεί τον υπολογισμό των παροχών παρατίθεται στο παράρτημα.

\subsection{Σχήμα ολοκλήρωσης Runge Kutta}\label{sub:rk}

Η αριθμητική επίλυση της συντηρητικής δ.ε λύνεται με το σχήμα Runge Kutta 4ης τάξης. Αρχικά, η εξίσωση \ref{eq:discrete} γράφεται ως:

\begin{equation*}
   f(U_i,t) = \dpart{\overline{U}_i}{t} = \overline{Q}_i -\dfrac{F_{i+1/2}-F_{i-1/2}}{\Delta x}
\end{equation*}

Και το διάνυσμα των συντηρητικών μεταβλητών για τον κόμβο i και για κάθε βήμα του αλγορίθμου k είναι:

\begin{equation}
    U^k_i = U_{t-1} + dt\cdot C_k \cdot f(U^{k-1}_i, t-1)
    \label{eq:rk}
\end{equation}

Η τιμή της επόμενης χρονικής στιγμής t είναι η $U_4$ δηλαδή για k=4. Οι τιμές του συντελεστή $C_k$ είναι: $C_k = [0.1084, 0.2602, 0.5052, 1]$. Οι τιμές που υπολογίζονται στα ενδιάμεσα βήματα συνεισφέρουν αποκλειστικά στη νέα τιμή της $f(U_i, t)$ παρά μόνο την περίπτωση του τελευταίου βήματος, όπου ανανεώνεται η τιμή του διανύσματος για το επόμενο χρονικό βήμα.

\subsection{Ορισμός οριακών συνθηκών} 

Αρχικά, αναφέρεται πως έχουμε 2 εικονικά κελλιά (ghost cells) τα οποία βρίσκονται μπροστά απο το πρώτο κελλί του χωρίου και μετά το τελευταίο. Σε αυτά τα κελλιά ορίζουμε τις οριακές συνθήκες εισόδου και εξόδου και το σχήμα ανακατασκευής του Roe είναι υπεύθυνο να ανανεώσει τις τιμές των παροχών στο πρώτο και το τελευταίο face με βάση τις τιμές στα ghost cells. 

Εκτός απο τις τιμές στα ghost cells που καθορίζονται απο τα δεδομένα στο αεροφυλάκιο, αρχικοποιήσαμε τις τιμές των κελλιών ως εξής: Θεωρώντας πως το αεροφυλάκιο είναι απομονωμένο απο τον αγωγό πρίν απο τη στιγμή t=0, όλα τα κελλιά πλήν του πρώτου ghost cell, αρχικοποιούνται με τις τιμές του περιβάλλοντος (εξόδου) για την πυκνότητα και την πίεση και με μηδενική ταχύτητα.

Επιπλέον, για τις τιμές που δεν έχουμε δεδομένες σε είσοδο και έξοδο, (ταχύτητα σε είσοδο έξοδο, πυκνότητα στην έξοδο), οι τιμές στα ghost cells ανανεώνονται ωστε να είναι ίσες με τις τιμές του γειτονικού κελλιού (μηδενική κλίση).

Τέλος, αναφέρεται πως προφανώς όλες οι οριακές συνθήκες ορίσθηκαν για τις πρωτογενείς μεταβλητές.

Οι ακριβείς τιμές ακολουθώντας τα προσωπικά δεδομένα παρατίθενται παρακάτω στον πίνακα \ref{tab:data}.

\subsecion{Αλγόριθμος επίλυσης}

Ο αλγόριθμος επίλυσης υλοποιώντας τις μεθόδους που περιγράφηκαν παραπάνω φαίνεται σχηματικά στο σχήμα \ref{fig:flowchart}.


\setlength{\fboxrule}{2pt} % Border thickness
\setlength{\fboxsep}{15pt} % Space between content and border

\begin{figure}[hbt]
    \centering
    \tikzstyle{block} = [rectangle, draw, text width=20em, text centered, rounded      corners, minimum height=3em]
    %/pgf/arrow keys/length=2mm
    \fbox{%
        \begin{tikzpicture}
             [node distance=1.45cm,
             start chain=going below,]
            \node (n1) at (0,0) [block]  {Αρχικοποίηση μεταβλητών};
            \node (n2) [block, below of=n1] {Δημιουργία πλέγματος};
            \node (n3) [block, below of=n2] {Υπολογισμός εμβαδού διατομής};
            \node (n4) [block, below of=n3] {Ορισμός συνοριακών συνθηκών};
            \node (n5) [block, below of=n4] {Υπολογισμός συντηρητικών μεγεθών};
            \node (n6) [block, below of=n5] {Υπολογισμός παροχών στα faces};
            \node (n7) [block, below of=n6] {Αποθήκευση προηγούμενων τιμών (cons)};
            \node (n8) [block, below of=n7] {Βήμα RK - υπολογισμός νέων συντηρητικών μεταβλητών};
            \node (n9) [block, below of=n8] {Ανανέωση πρωτογενών μεταβλητών};
            \node (n10) [block, below of=n9] {Ανανέωση συνοριακών συνθηκών};
            \node (n11) [block, below of=n10] {Ανανέωση συντηρητικών μεταβλητών - υπολογισμός νεων παροχών};
            \node (n12) [block, below of=n11] {Επόμενο χρονικό βήμα};
            \node (n13) [block, below of=n12] {Τέλος προσομοίωσης};
            % Connectors
            \draw [->] (n1) -- (n2);
            \draw [->] (n2) -- (n3);
            \draw [->] (n3) -- (n4);
            \draw [->] (n4) -- (n5);
            \draw [->] (n5) -- (n6);
            \draw [->] (n6) -- (n7);
            \draw [->] (n6) -- (n7);
            \draw [->] (n7) -- (n8);
            \draw [->] (n8) -- (n9);
            \draw [->] (n9) -- (n10);
            \draw [->] (n10) -- (n11);
            \draw [->] (n11) -- (n12);
            \draw [->] (n12) -- (n13);
            \draw [->] (n11.east) -| ++(1,0) |- (n8.east);
            \draw [->] (n12.west) -| ++(-1,0) |- (n7.west);
        \end{tikzpicture}
    }
    \caption{Αλγόριθμος επίλυσης μοντέλου}
    \label{fig:flowchart}
\end{figure}
